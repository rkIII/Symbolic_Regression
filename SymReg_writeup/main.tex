
%%%
% Any line that begins with a percent symbol is a comment. To compile
% this document and view the output:
%
% Run Latex
% Run Bibtex
% Then run Latex twice.
%
% This should produce the output PDF file named main.pdf
%%%

% This defines the style to use for this document.
% Do not modify.
\documentclass[letterpaper]{article}

% The following are akin to "import" statements in Python or Java -
% these import useful commands into the document for you to use.  You
% don't have to modify any of these lines. The AAAI package formats
% this document in the style of submissions to the American
% Association for Artificial Intelligence conference, one of the top
% AI conferences in the world. You will find that many academic
% publications in AI use this format.
\usepackage{aaai} 
\usepackage{times} 
\usepackage{helvet} 
\usepackage{courier} 
\setlength{\pdfpagewidth}{8.5in} 
\setlength{\pdfpageheight}{11in} 
\usepackage{amsmath}
\usepackage{amsthm}
\usepackage{graphicx}
\usepackage{graphics}
\usepackage{moreverb}
\usepackage{array,booktabs}
\newcolumntype{L}{@{}>{\kern\tabcolsep}l<{\kern\tabcolsep}}
\usepackage{subfigure}
\usepackage{epsfig}
\usepackage{txfonts}
\usepackage{palatino}
\usepackage{algpseudocode}
\usepackage{multirow, multicol}
\usepackage{url}
\usepackage{tablefootnote}
\usepackage{color}

\setcounter{secnumdepth}{1}
\nocopyright

% Fill in your paper title, names and emails below
% The "\\" is used to break lines. The \url command
% is useful for typesetting URLs and email addresses (it uses the
% Courier font).
\title{Using Genetic Programming to Implement Symbolic Regression}
 \author{Caroline Thompson \and Rich Korzelius\\
 \url{{cathompson, rikorzelius}@davidson.edu}\\
 Davidson College\\
 Davidson, NC 28035\\
 U.S.A.}

% This is the "true" start of the document. All the text in your
% write-up should be placed within the \begin{document} and
% \end{document} decorators.
\begin{document}

\maketitle % formats the title nicely, do not modify

% While at this point you could just begin your write-up, often, it's
% useful to write each section of your write-up in a separate tex
% file (not unlike the modular decomposition you do for code you
% write). These \input commands insert the contents of the
% specified tex files in the order specified. Every write-up you
% submit must contain the following sections, in the shown order. Open
% each of the indicated tex files to understand what goes in each
% section, as well as for more TeX tips.



\begin{abstract}



We use genetic programming to implement symbolic regression. Specifically, we use symbolic regression to determine a mystery function f. We do this by creating many potential functions
and use genetic programming to find the best fit to the mystery function. We are still awaiting results to find the functions that best fit our data points. 

\end{abstract}




\section{Introduction}
\label{sec:intro}



Genetic programming is a useful artificial intelligence problem because it enables computers to automatically find the most optimal solution to a problem using the evolutionary idea of natural selection and survival of the fittest.
In this problem, we create a computer program to determine a mystery
function by using symbolic regression. To do this, we use genetic programming
to search for the best function that matches the mystery function. We assign each individual a fitness value and use that to 
determine which individuals survive for the next generation. 
To weed out weak individual functions and improve the optimality of each subsequent generation, we probabilistically select fit individuals to participate in three methods to produce the next generation: reproduction, crossover, and mutation.  \\

We use methods such
as reproduction, crossover and mutation to improve the fittest functions, as described by John Koza \cite{genetic programming}. Although we use LEAST SQUARES?? to determine each individual's fitness value, other literature suggests that combining this method with linear scaling can produce better results \cite{linearscaling}. In future work, we would include this method in our experiments ad see how the output compares to our original results. Another issue that could hinder our results is premature convergence, in which the computer may cease searching for a solution before the optimal solution is determined. James Edward Baker suggests the use of adaptive selection methods such as a fixed selection algorithm or hybrid systems to prevent this problem \cite{selectionmethods}. 

We are still working on our results, which will be briefly summarized in this section. 

In the following sections, we discuss the background of this problem, our experiment, and our results. 





\section{Background}
\label{sec:background}



We later describe how we execute the reproduction, mutation, and crossover methods to produce generations, but we use a method from Brad Miller and David Goldberg \cite{tournamentselection} called tournament selection to select individuals. In this method, we choose a sample of the population and select the winner to be the individual with the best fitness value. We insert this individual into a "mating pool" of other tournament winners. Since the average fitness value of the mating pool is better than that of the general population of individuals, we can use this difference to calculate the selection pressure, i.e. the amount that fitter individuals are preferred. Once a random subset of the population is chosen, we cycle through the sampled individuals looking for the one with the best fitness score (least error) and select that individual for crossover. This ensures that subsequent generations have better average fitness values and that more fit functions have a better chance at being chosen then lesser functions. The idea behind tournament selection is that trees with worse fitness scores will always be weeded out through the tournament process, resulting in stronger parents being chosen for crossover.






\section{Experiments}
\label{sec:expts}

\textbf{Experiment 1}
To find a mystery function using genetic programming, we first generate an initial population of possible functions. To do this, we randomly
create functions expressed by the arithmetic operators +, -, *, /, positive integer powers of x, and
integer constants. We choose an initial population size of 300 individuals and a maximum depth of 10 for each individual tree. We choose these parameters because they are large enough for us to get a large variety of possible functions while maintaining a certain degree of randomness, but small enough to run quickly and prevent memory issues. We only use integers from -5 to 5 for the same reasons. To prevent the possibility of functions calculating negative exponents, we take the absolute value of each integer power of x. When creating the initial population, each operator has an equal chance at being chosen for each equation. Thus, we have a wide variety of different functions. \\

Second, we give each function a fitness value to determine how closely its value matches the mystery function's value. In our case, the fitness value for an individual is its residual error compared to the actual dataset. It follows that individuals would prefer to have lower fitness scores (i.e., for us lower fitness scores are better). For each individual, we evaluate the function value for a given x, and subtract this from the mystery function value of x. Then, we take the absolute value of the difference to find the error. For the purposes of this experiment, we were able to work with lower fitness values being better thanks to the tournament selection algorithm. We use the fitness value to determine the fittest individuals and choose individuals for crossover, mutation, and reproduction using tournament selection. Our regression runs for a set number of generations\\

Next, we select individuals for reproduction, mutation, and crossover. We decide on a 1 percent reproduction rate and 10 percent mutation rate. We arbitrarily chose the 1 percent reproduction rate, and converged on the 10 percent mutation rate as a way to ensure a higher level of diversity amongst our generations. For reproduction, we pick the fittest 1 percent of the population and clone them into our next generation. Rather than simply selecting the fittest individuals for mutation and crossover, we select individuals proportionately to their fitness values using tournament selection, as described previously. This prevents possible diversity issues from arising because although fitter individuals are more likely to be chosen, less-fitter individuals also have a chance at being chosen. The first operation to be executed is reproduction. The next operation is mutation. We select 10 percent of the population using tournament selection and then randomly choose a node in each tree to mutate. We choose to mutate terminals with terminals and operators with operators to maintain some structure. If the mutation causes and mathematical inconsistencies (i.e., negative exponent, divide by 0, raising something to the x power directly) we adjust for those situations and mend the tree. Finally, we execute the crossover operation. In this method, we select two individuals and randomly choose either the right or left subtrees in each individual. Then, we swap the trees at these points in the two trees we selected.  By selecting subtrees in this way we maintain some stable structure throughout the process.\\

After executing these operations a number of times, the trees get fitter and the fitness values get closer to zero, so we get closer to finding the mystery function with each generation. We choose to operate over 30 generations. We felt that this number gave us a good chance of convergence but avoided overfitting.\\

\textbf{Experiment 2}
We are also tasked with performing symbolic regression on a file of 100,000 data points with measured values of three x-values. Although we are unable to produce cohesive results, we explain how we would like to execute this regression. We continue to use the same parameters as in the first regression. However, to prevent overfitting, we separate the data file into a test file with 20,000 data points and a training file with 80,000 points. Next, we calculate the fitness values of the individuals in the training set and determine the 10 fittest individuals. We then use these individuals with the data from the test set to find the optimal solution by performing the evolution method on a population of those 10 fittest trees. 

For the experiment with vector functions of 3 variables, we choose to keep our reproduction, mutation, and crossover functions the same as in the first experiment. We also maintained the same population size, terminal range, and number of generations. We change the maximum depth of the tree to 5 and make all randomly generated new trees in the initialization process to be of this length. This makes the process of evaluating and parsing trees easier and ensures a variety of the three variables ($x_1, x_2, x_3$) appear in the functions.








\section{Results}
\label{sec:results}

\textbf{Experiment 1}
For the first experiment, we were able to run multiple trials in which each trial went through the same number of evolutions. As you can see from the table below, we ran ten different regressions using our genetic algorithm and it resulted in a noticeable large range of residual errors ($|y_{actual} - y_{calculated}|$). Each trial and a different level of convergence after the specified number of iterations. One factor that we believe contributed to this fluctuation is the range of values chosen for evaluation. The dataset generated by Generator 1 had more of the function activity occurring between [0,2] with a local maximum within that range. The rest of the data maintained fairly consistent and did not fluctate much. This resulted in solutions of horizontal lines of lines with small slopes. \\

Another hypothesis for the lack of consistency is a lack of diversity within the generations. A lack of diversity could result in crossover operations that don't drastically change the structure or output of a function, resulting in generations with average fitness scores very similar to that of their predecessors.

\begin{minipage}{\linewidth}
\centering
\captionof{table}{Experiment 1 - Mystery Function of Generator 1} \label{tab:title} 
\begin{tabular}{ C{1.25in} C{.85in} *4{C{.75in}}}\toprule[1.5pt]
\bf Regression & \bf Solution & \bf Error (Fitness) \\midrule
1        &  ((-1 - x) + 3)     & 11.5942028986 \\
2        &  (3 / (x * x))     & 24.9275362319 \\
3        &  ((-1 + x) ^ 5)   & 89.9623475181  \\
4        &  ((2 / x) ^ 2)     & 15.0724637681 \\
5        &  (x / ((-3 ^ 3) + 5))     & 82.3451910408 \\
6        &  (5 + 5)     & 55.0724637681  \\
7        &  (x + (-2 ^ x))     & 11.3224637681 \\
8        &  (-1 + x)     & 55.0724637681 \\
9        &  ((-4 / (-5 + (2 + x))) ^ 3)     & 19.9275362319  \\
10        &  ((x / (3 - x)) - x)     & 19.9275362319  \\
\bottomrule[1.25pt]
\end {tabular}\par
\bigskip
The regression data from Experiment 1.
Population size = 500
Max Tree Depth = 10
Number of Generations per Trial = 10
\end{minipage}

\textbf{Experiment 2}
Unfortuantely, we were not able to successfully compute the regression using the dataset from the file data.txt. Our program (which is still running) hangs and we continue to await results. 

However, we chose to at least include data from our test runs to examine the performance of the algorithm over a smaller set. For our test we created sets of data points with values of $x_1, x_2, x_3$ ranging from [1,10]. We then set arbitrarily assigned y values. the data was put in parallel arrays. Below are our results.

\begin{minipage}{\linewidth}
\centering
\captionof{table}{Experiment 2 - Mystery Vector Function} \label{tab:title2} 
\begin{tabularx}{\linewidth}{@{} C{1in} C{.85in} *4X @{}}\toprule[1.5pt]
\bf Regression & \bf Solution & \bf Error (Fitness) \\\midrule
1        &  (((((-2 ^ 3) * (2 * x_2)) / ((-4 - x_2) * (5 / 4))) - (((x_1 + -4) ^ (-1 ^ 1)) * ((-2 ^ -4) / (x_3 * x_2)))) * 0)     & error: 359.0 \\
2        &  (((((x_2 - x_2) - (x_3 - x_1)) - ((1 * 0) - (x_1 * x_1))) + (((x_1 * x_3) + (-1 - x_2)) + ((x_1 * x_2) / (x_1 * x_3)))) ^ ((((1 / x_1) / (-4 - x_3)) / ((-2 / x_3) * (3 ^ -4))) / (((3 ^ 4) ^ (x_3 - x_1)) - ((-5 + -1) ^ (-4 ^ 1)))))     & error: 351.0 \\
3        &  (((((x_2 ^ -1) - (2 ^ 2)) * ((x_1 / x_2) / (x_1 * -1))) * (((-2 ^ -3) + (1 - x_2)) / ((-4 - x_3) ^ (0 ^ 5)))) / ((((x_2 - -4) ^ (x_3 ^ -3)) / ((3 - -3) * (5 * -5))) - (((-4 + -5) / (x_3 + x_1)) - ((-3 / -4) * (-2 ^ -2)))))
error: 332.04405746    & error: 332.04405746 \\
\bottomrule[1.25pt]
\end {tabularx}\par
\bigskip
Results from our tests with Symbolic Regress of vector functions of three variables.
$x_1 = [1.0, 2.0, 3.0, 4.0, 5.0, 6.0, 7.0, 8.0, 9.0, 10.0]
x_2 = [6.0, 4.0, 5.0, 8.0, 9.0, 7.0, 2.0, 3.0, 10.0, 1.0]
x_3 = [9.0, 10.0, 5.0, 6.0, 8.0, 2.0, 1.0, 3.0, 7.0, 4.0]
y = [-2.0,1.0,6.0,13.0,22.0,33.0,46.0,61.0,78.0,97.0]$
\end{minipage}

\section{Conclusions}
\label{sec:concl}

Although our experiment did not yield ideal results, we were able to make some conclusions about the effectiveness of the algorithm and comment on its efficiency. We were unable to converge on unique and distinct solutions for either problem. We believe that the genetic algorithm could be very effective if the proper design and constraints are used based on the problem we are trying to address. In our case, modifying the standard crossover function or our selection process could have resulted in better parent functions. This could lead to a consistent decrease in average error from generation to generation. Then, all that is needed is the correct number of evolutions to reach the desired level of accuracy (too avoid overfitting). In further studies, we would expand our function span to include trigonometric functions, non-integer powers of x, and non-integer constants. We would also like to experiment with different selection methods, such as hybrid systems to prevent premature convergence, as mentioned previously in this paper. Adapting our Symbolic Regression solver to better manage the population and more easily maintain diversity and evolutionary improvement. 

\section{Acknowledgements} 
\label{sec:ack} 

Thanks to Davidson CS Department!


% This creates the references section. Open the project1.bib file to
% see how to organize your references.
\bibliography{project1}
\bibliographystyle{aaai} % sets citation and bib style, do not modify

\end{document}
