
\section{Experiments}
\label{sec:expts}


We seek to find a mystery function using genetic programming. \\

First, we generate an initial population of possible functions. To do this, we randomly
create functions expressed by the arithmetic operators +, -, x, /, positive integer powers of x, and
integer constants. We choose an initial population size of 300 possible functions. We also choose a maximum depth of 10 for each function tree. We choose these parameters because they are large enough for us to get a large variety of possible functions, but small enough to run quickly and prevent the computer from running out of memory. We only use integers from -5 to 5 for the same reasons. To prevent the function calculating negative exponents, we take the absolute value of an integer power of x. When creating the initial population, each operator has an equal chance at being chosen for each equation. Thus, we have a wide variety of different functions. \\

Second, we give each function a fitness value to determine how closely its value matches the mystery function's value. For each individual, we evaluate the function value for a given x, and subtract this from the mystery function value of x. Then, we take the absolute value of this function and assign it as the individual's fitness value. We use the fitness value to determine the fittest individuals and choose individuals for crossover, mutation, and reproduction proportionately to their fitness values. Our program runs until an individual's fitness is as close to zero as possible, which indicated that we have found the mystery function.\\

Next, we select individuals for reproduction, mutation, and crossover. We decide on a 1 percent reproduction rate, 10 percent mutation rate, and 89 percent crossover rate. Rather than simply selecting the fittest individuals for these operations, we select individuals proportionately to their fitness values. This prevents possible diversity issues from arising because although fitter individuals are more likely to be chosen, less-fitter individuals also have a chance at being chosen. As described previously, we use a tournament selection method to determine the best individuals for the operations. The first operation to be executed is reproduction. In this method, we select an individual and simply copy it into the next generation. The next operation is mutation. We select an individual and then randomly choose a node in the tree to mutate. Once we select a node, we delete the subtree after that point and reconstruct another tree using the same method we use to generate the initial population. Finally, we execute the crossover operation. In this method, we select two individuals and randomly choose either the right or left subtrees in each individual. Then, we swap the trees at these points in the two trees we selected. \\

After executing these operations a number of times, the trees get fitter and the fitness values get closer to zero, so we get closer to finding the mystery function with each generation.




